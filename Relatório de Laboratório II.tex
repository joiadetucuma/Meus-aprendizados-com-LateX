\documentclass[a4paper, 12pt]{article} 
\usepackage[utf8]{inputenc} 
\usepackage[brazil]{babel} 
\usepackage[lmargin=3cm,tmarging=3cm,rmargin=2cm,bmargin=2cm]{geometry} 
\usepackage[T1]{fontenc} 
\usepackage{amsmath,amsthm,amsfonts,amssymb,dsfont,mathtools,blindtext} 
\usepackage{amsmath, calc, xcolor}
\usepackage{blindtext} 
\usepackage{graphicx} 
\usepackage[round]{natbib}
\bibliographystyle{dinat}


\title{Relatório de Laboratório II}
\author{Maria Clara Rodrigues Miranda}
\date{03 de Novembro 2022}

\begin{document}
\maketitle

\section{Assunto}
O experimento \textbf{Análise Gráfica de Dados}, foi executado através de materiais específicos para medir o comportamento da mola quando sofre influencia de um peso. O procedimento, as medidas e os resultados desse experimento foram obtidos pela aluna \textit{Maria Clara Rodrigues Miranda da turma de Física - Bacharelado, FB01.}

\section{O Experimento}
Para o esse procedimento foram utilizados: uma mola de 9mm de diâmetro, um porta-peso de 10 g, cinco pesos de 50g, um suporte e uma régua milimetrada. A partir disso, iremos estudar a influência dos pesos acoplados a extremidade da mola e elabora um gráfico desses valores.

\section{O Objetivo}
O objetivo desse experimento é calcular a constante elástica, comparar com a literatura, de uma mola através da análise dos resultados obtidos através da interpretação gráfica feita a partir desses mesmos resultados.

\section{A Teoria}
Neste experimento usamos o princípio da Lei de Hooke. Formulada pelo cientista Robert Hooke na década de XV, essa lei afirma que quando uma força externa é aplicada na mola, ela é capaz de deformar a mola fazendo-a produzir uma força contrária a força externa, o que chamamos de força elástica e essa força é diretamente proporciona a deformação da mola. Onde Fel = força elástica em Newtons, k = constante elástica e x = deformação da mola. Essa fórmula é:
\begin{equation}
    Fel = -k \times x
\end{equation}

\begin{figure}[!ht]
\centering 
\includegraphics[width=7cm]{pictures/lei de hooke.png} 
\caption{Lei de Hooke} 
\label{Fig01} 
\end{figure}

\pagebreak
\section{Os Procedimentos}
Primeiro, encaixamos a mola no suporte e com o auxílio da régua marcamos o nosso ponto referencial, que é o nosso "marco zero", caracterizando o começo do experimento. Em seguida, encaixamos o porta-peso de 10g mais um peso de 50g (totalizando 60g) e marcamos com a régua qual foi a medida em milímetros desse desse em relação a medida anterior. Realizamos essa segunda parte do procedimento cinco vezes, adicionando a cada 50g, mais 50g.

\section{Os Resultados}
\subsection{Passo 1}
Sobre a adição de pesos em cada etapa, foram obtidas seis medidas, incluindo o ponto referencial. A variação da primeira medida (ponto referencial) para a segunda foi de 30mm, da segunda para a terceira = 24mm, da terceira para a quarta = 24mm, da quarta para a quinta = 25mm e da quinta para a sexta = 25mm. A massa dos resultados foram convertidas de grama para quilograma e de milímetro para metro. São elas:

\begin{table}[!ht]  
    \centering
    \begin{tabular}{|c|c|}
    \hline

    m \pm \Delta m & y ± \Delta y \\ \hline 
        0 & 7,0 x 10 ^ {-1} \pm 1 \\ \hline
        6,0 x 10 ^ {-2} \pm 1 & 6,7 x 10 ^ {-1} \pm 1 \\ \hline
        1,1 x 10 ^ {-1} \pm 1 & 6,46 x 10 ^ {-1} \pm 1 \\ \hline
        1,6 x 10 ^ {-1} \pm 1 & 6,22 x 10 ^ {-1} \pm 1 \\ \hline
        2,1 x 10 ^ {-1} \pm 1 & 5,97 x 10 ^ {-1} \pm 1 \\ \hline
        2,6 x 10 ^ {-1} \pm 1 & 5,92 x 10 ^ {-1} \pm 1 \\ \hline
 
    \end{tabular}
    \caption{\label{tab:table-name}Tabela de Resultados.}
\end{table}

\subsection{Passo 2}
Em seguida, aplicamos esse valor na fórmula da força peso:
\begin{equation}
    P = m \times g
\end{equation}

Onde:

m = massa em quilograma \vspace{1} 

g = gravidade = 9,8 metros por segundo ao quadrado \vspace{1}

E obtivemos os seguintes valores: \vspace{1}


\begin{table}[!ht]
    \centering
    \begin{tabular}{|c|c|}
    \hline
        $m \pm \Delta m$ & força peso \\ \hline
        $0$ & $0$ \\ \hline
        $6,0 x 10 ^ {-2} \pm 1$ & $5,88 x 10 ^ {-1}$  \\ \hline
        $1,1 x 10 ^ {-1} \pm 1$ & $1,078$ \\ \hline
        $1,6 x 10 ^ {-1} \pm 1$ & $1,568$ \\ \hline
        $2,1 x 10 ^ {-1} \pm 1$ & $2,058$ \\ \hline
        $2,6 x 10 ^ {-1} \pm 1$ & $2,548$ \\ \hline 
    \end{tabular}
\end{table}

\subsection{Passo 3}
Por fim, aplicamos esse resultado na fórmula da Lei de Hooke, substituindo Fel por P, e x por $\Delta$Y e obtivemos os seguintes valores:

\begin{table}[ht]
\centering
\begin{tabular}{|c|}
\hline
    $P = -k \times \Delta y$ \\ \hline
    $0  $                    \\ \hline
    $8,95 \times 10 ^ {-4}$  \\ \hline
    $1,70 \times 10 ^ {-3}$  \\ \hline 
    $2,57 \times 10 ^ {-3}$  \\ \hline
    $3,51 \times 10 ^ {-3}$  \\ \hline
    $4,54 \times 10 ^ {-3}$  \\ \hline
  \end{tabular}
\end{table}

Somando todos esses valores e comparando com a constante normal que nos foi fornecida durante o experimento de $20 N/m ^ {2}$, obtivemos o valor de $21 N/m ^ {2}$ (o que se encontra dentro da méia tendo em vista que a propagação de incerteza é de $\pm$1. Lembrando que a constante elástica é o que caracteriza a deformação da mola e condiz com a força necessária para que a mola sofra deformidade. As molas que apresentam grandes constantes são mais difíceis de serem deformadas, diferente das menores. 


\section{Conclusão}
Concluindo, então, que a reta formada pelo gráfico é crescente e que os valores são diretamente proporcionais uns aos outros e atingindo a meta do experimento: utilizar fórmulas para encontrar valores que sejam aplicáveis a Lei de Hooke e, principalmente, construir um gráfico e analisá-lo baseado nos resultanos obtidos.
Logo, finalizando o experimento, identificando os valores e aplicando corretamente todas as fórmulas, temos os dados para construir o gráfico de Hooke. Quando o eixo Y equivale ao P(Fel) e o eixo X equivale a Massa (m) o gráfico fica da seguinte maneira:

\begin{figure}[!ht]
\centering 
\includegraphics[width=12cm]{pictures/gráfico de hooke.jpg} 
\caption{Gráfico} 
\label{Fig02} 
\end{figure}


\pagebreak
\section{Referências}
NUSSENZVEIG, Herch Moysés. \textbf{Curso de física básica: Mecânica (vol. 1)}. Editora Blucher, 2013. \vspace{1}

MARQUES DA SILVA, Domiciano Correa. \textbf{“Representação Gráfica Da Lei de Hooke. O Gráfico Da Lei de Hooke.”} PrePara Enem, www.preparaenem.com/fisica/ \vspace{1}

HELERBROCK, Rafael. “Lei de Hooke: Conceito, Fórmula, Gráfico, Exercícios.” Brasil Escola, brasilescola.uol.com.br/fisica/lei-de-hooke.htm.
‌

\end{document}

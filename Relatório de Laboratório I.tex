\documentclass[a4paper, 12pt]{article} 
\usepackage[utf8]{inputenc} 
\usepackage[brazil]{babel} 
\usepackage[lmargin=3cm,tmarging=3cm,rmargin=2cm,bmargin=2cm]{geometry} 
\usepackage[T1]{fontenc} 
\usepackage{amsmath,amsthm,amsfonts,amssymb,dsfont,mathtools,blindtext} 
\usepackage{amsmath, calc, xcolor}
\usepackage{blindtext} 
\usepackage{graphicx} 
\usepackage[round]{natbib}
\bibliographystyle{dinat}


\title{Relatório de Laboratório I}
\author{Maria Clara Rodrigues Miranda}
\date{25 de Outubro 2022}

\begin{document}
\maketitle

\section{Assunto}
O experimento \textbf{Medidas Físicas}, foi executado através de materiais específicos para medir objetos de diversos tamanhos, nesse caso, foi utilizados um objeto a ser medido e três objetos para medição de medidas. O procedimento, as medidas e os resultados desse experimento foram obtidos pela aluna \textit{Maria Clara Rodrigues Miranda da turma de Física - Bacharelado, FB01.}

\section{O Experimento}
Para o esse procedimento foram utilizados: régua, paquímetro, micrômetro e uma esfera de aço. Através desses três objetos de medida, iremos medir a esfera em cada um deles, anotar seu tamanho e medir o seu volume. 


\begin{figure}[ht]
\centering 
\includegraphics[width=7cm]{régua.jpg} 
\caption{Régua} 
\label{Fig01} 
\end{figure}

\begin{figure}[ht]
\centering 
\includegraphics[width=7cm]{paquimetro.png} 
\caption{Paquímetro} 
\label{Fig02} 


\end{figure}

\begin{figure}[ht]
\centering 
\includegraphics[width=7cm]{micrometro.png} 
\caption{Micrômetro} 
\label{Fig03} 
\end{figure}


\section{O Objetivo}
O objetivo desse experimento é aprender a manusear os materiais de medida do laboratório, aplicar fórmulas e aprender as margens de erro através da fórmula do volume de uma esfera e de interpretação. Logo, o objetivo do experimento é fomentar o aprendizado através de práticas de laboratório, relembrar e aplicar o uso de fórmulas de esferas e medidas usadas no ensino fundamental/médio.

\section{A Teoria}
O sistema de medidas decimais surgiu em meados do século XVI durante a Revolução Francesa a fim de unificar e acabar com as dificuldades de comércio e da indústria. Entre 1795 e 1982, ocorreram muitas mudanças nos conceitos e leis de medição, até que, em 1983, através da ciência, foram feitas a últimas alterações nas definições de medidas. Hoje em dia, as unidades de medidas estão em todo lugar e regulam os padrões de comércio e indústria, cumprindo a necessidade que houve durante a Revolução Francesa e ajudando as necessidades do mundo atual.

\section{Os Procedimentos}
Primeiramente, com a régua e o auxílio que um pequeno suporte acoplado a ela, encaixamos e esfera de aço e concluímos que o seu diâmetro é de 26,50±0,5 mm. O mesmo procedimento foi feito com um paquímetro (usado para medir dimensões lineares internas e externas e profundidade de uma peça, geralmente de tamanho pequeno) e o micrômetro (usado para medir altura, espessura, largura e profundidade de um objeto em milímetros ou polegadas) e concluímos que suas medidas são 27,10±0,05 mm e 26,918±0,004 mm respectivamente. A diferença de medida dos três objetos se deu por conta de serem objetos diferentes com meios de medir distintos. 

\section{Os Resultados}
Para encontrar o raio, a fim de aplicar na fórmula do volume da esfera e concluir o experimento, é necessário dividir o diâmetro de cada um dos resultados por dois. Logo, os resultados obtidos e aplicados na fórmula:

\begin{equation}
    V = \frac{4}{3}\pi r^{3}
\end{equation}

Tendo em vista que, das fórmulas de propagação de incertezas, a que mais se assemelha a fórmula do volume é:

\begin{equation}
    w = ax ^{p}. y^{q}
    
\end{equation}

Onde:

w = Volume \vspace{1}

a = \frac{4}{3} \vspace{1}

x = pi \vspace{1}  
   
p = expoente de pi \vspace{1}   

y = raio \vspace{1}

q = expoente do raio \vspace{3}

\vspace{2}
Após isso, é necessário calcular a propagação de incerteza do volume, através da fórmula:

\begin{equation}
     \left( \frac{\delta w}{w} ^{2} \right) = \left (p \frac{\delta x}{x} ^{2} \right) + \left( q \frac{\delta y}{y} ^ {2} \right)
\end{equation}  

Logo, a interpretação dos resultados fica: 
\begin{table}[h]
\centering
\begin{tabular}{|l|l|l|l|}
\hline
Instrumento & (D± \Delta D) & (r± \Delta r ) & (V±\Delta V)   \\ \hline
Régua       & 26,50±0,5  & 13,25±0,5   & 9743,97±3488 mm ^ {3}   \\ \hline
Paquímetro  & 27,10±0,05  & 13,55±0,05  & 10420,93±115 mm ^ {3} \\ \hline
Micrômetro  & 26,918±0,004 & 13,46±0,004 & 10191,9±527 mm ^ {3}  \\ \hline

\end{tabular}
\end{table}

\section{Conclusão}
Por fim, o experimento foi concluído com sucesso e todos os resultados foram obtidos com êxito. Retomando, o objetivo do experimento era: calcular a partir de objetos distintos o diâmetro e o raio de uma esfera de aço e aplicar na fómula de volume da esfera e aprender sobre a propagação de incertezas


\end{document}

